\documentclass[lighthipster]{cvtheme}
\usepackage[T1]{fontenc}
\usepackage[utf8]{inputenc}
\usepackage[spanish]{babel}
\usepackage[default]{raleway}
\usepackage[margin=1cm, a4paper]{geometry}

\begin{document}
\thispagestyle{empty}
\section*{}

\simpleheader{Fernando Esteban Fêtis Riquelme}{Magister en Data Science | Ingeniero Matemático}

\subsection*{}
\vspace{4em}

\setlength{\columnsep}{1.5cm}
\columnratio{0.23}[0.75]

\begin{paracol}{2}

\paracolbackgroundoptions
\footnotesize
{\setasidefontcolour
\flushright
\vspace{1cm}
\begin{center}
    \roundpic{images/fernando.png}
\end{center}

\vspace{0.8cm}
\bg{cvgreen}{white}{Acerca de mí}\\[0.5em]
Soy una persona apasionada por la inteligencia artificial y el aprendizaje continuo, por lo que siempre busco nuevos desafíos que me permitan aplicar mis habilidades, aprender y crecer profesionalmente. Hace poco creé una \href{https://fernandofetis.notion.site}{página personal} para compartir algunos tópicos de IA generativa.

\vspace{0.8cm}
\bg{cvgreen}{white}{Objetivos a corto plazo}\\[0.5em]
Este segundo semestre estaré enfocado principalmente en el estudio e investigación de los modelos de difusión y flow matching.

\vspace{0.8cm}
\bg{cvgreen}{white}{Objetivos a mediano plazo} \\[0.5em]
Mi principal objetivo a mediano plazo es ingresar a un programa de doctorado en el campo de la inteligencia artificial con el fin de poder contribuir activamente al avance de esta disciplina.

\vfill
\flushleft{
    \infobubble{\faLanguage}{cvgreen}{white}{Nivel B2 TOEFL ITP}
    \infobubble{\faAt}{cvgreen}{white}{fernando.fetis@uchile.cl}
    \infobubble{\faPhone}{cvgreen}{white}{+56 9 4102 2335}
}

}
\switchcolumn\small

\section*{Formación académica}
\vspace{0.5cm}

\begin{tabular}{r| p{0.5\textwidth} c}
    \cvevent{2017--2023}{Ingeniería Civil Matemática}{Universidad de Chile}{fcfm \color{cvred}}{Realicé mi pregrado en el programa de ingeniería matemática en Beauchef, donde me especialicé fuertemente en temas de aprendizaje automático e inteligencia artificial. Este programa lo terminé con distinción máxima.}{uchile.png} \\
    
    \cvevent{2023--2024}{Magister en Ciencia de Datos}{Universidad de Chile}{fcfm \color{cvred}}{Continué mi especialización entrando al magíster en data science de la misma facultad. En este programa me especialicé fuertemente en temas de IA generativa, concluyendo con un \href{https://github.com/fernando-fetis/mds-thesis}{trabajo de tesis} acerca de los puentes de Schrödinger, el cual es un paradigma reciente que generaliza a los modelos de difusión. Este programa lo terminé con distinción máxima.}{uchile.png}
\end{tabular}

\section*{Antecedentes laborales}
\vspace{0.5cm}

\begin{tabular}{r| p{0.55\textwidth} c}

    \cvevent{2025}{Profesor adjunto}{MDS - Universidad de Chile}{Santiago \color{cvred}}{Actualmente me encuentro dictando el curso \href{https://www.u-cursos.cl/ingenieria/2023/2/MDS7203/1/datos_curso/bajar_programa?id=150626&491897610}{Modelos Generativos Profundos}, asociado al programa de Magister en Ciencia de Datos de la Facultad de Ciencias Físicas y Matemáticas.}{uchile.png} \\

    \cvevent{2025}{Data scientist}{IDIA - Universidad de Chile}{Santiago \color{cvred}}{En la misma facultad estoy trabajando como data scientist en proyectos de la iniciativa.}{uchile.png} \\

    \cvevent{2025}{Investigador}{RIAL}{Santiago \color{cvred}}{También me encuentro trabajando como investigador en la startup RIAL, donde trabajamos con modelos de difusión para generación de imágenes de alta calidad.}{uchile.png} \\

    \cvevent{2023}{Investigación de redes neuronales basadas en la física}{Inria Chile}{Santiago \color{cvred}}{Durante mi segunda y tercera práctica profesional, formé parte de un equipo de investigación enfocado en aplicar redes neuronales a la simulación de sistemas dinámicos complejos guiados por una PDE. Contribuí a analizar y mejorar la precisión de estas técnicas, buscando soluciones innovadoras para la resolución de problemas en diversas áreas.}{inria.png}\\

    \cvevent{2023}{Búsqueda de arquitecturas neuronales para PINNs}{Inria Chile}{Santiago \color{cvred}}{Posterior a mi práctica profesional, continué mi trabajo en Inria, esta vez enfocado en la optimización de arquitecturas de redes neuronales basadas en la física (PINNs) mediante algoritmos genéticos guiados por procesos gaussianos. El trabajo concluyó con la creación de una librería de Python interna enfocada en PINNs.}{inria.png}\\

    \cvevent{2021}{Edición del apunte del curso de Machine Learning}{Universidad de Chile}{Santiago \color{cvred}}{Colaboré en la revisión y mejora del material educativo para el curso de \href{https://www.u-cursos.cl/ingenieria/2021/1/MA5204/1/datos_curso/bajar_programa?id=39073&2038924245}{Aprendizaje de Máquinas}, actualizando contenidos y corrigiendo errores. También añadí secciones clave que alineaban mejor el material con los objetivos del programa.}{uchile.png}
\end{tabular}

\vfill
\newpage

\end{paracol}

\thispagestyle{empty}

\setlength{\columnsep}{1.5cm}
\columnratio{0.01}[0.90]

\begin{paracol}{2}

\paracolbackgroundoptions
\footnotesize
{\setasidefontcolour
\flushright
$ $
}
\switchcolumn\small

\section*{Antecedentes laborales}
\vspace{0.5cm}

\begin{tabular}{r| p{0.7\textwidth} c}

    \cvevent{2020}{Simulación de sistema dinámico asociado a la extracción minera}{Centro de Modelamiento Matemático}{Santiago \color{cvred}}{Durante mi primera práctica profesional, desarrollé una interfaz gráfica en MATLAB para visualizar y analizar datos relacionados con la lixiviación in situ de un proyecto minero. Mi trabajo facilitó la interpretación de los datos, mejorando la toma de decisiones en el proceso de optimización operativa.}{cmm.png}\\

    \cvevent{2015}{Asistente en Olimpiadas de Matemáticas}{Universidad de La Frontera}{Temuco \color{cvred}}{Colaboré en la organización de la \textit{Olimpiada Regional de Matemática}, participando en la corrección de pruebas y en la creación de preguntas. Además, contribuí a la edición del libro de problemas de publicación anual.}{ufro.png}

\end{tabular}

\section*{Otros antecedentes}
\vspace{0.5cm}

\begin{tabular}{r p{0.85\textwidth} c}

    \cvbullet{Experiencia académica y docencia}{\vspace{-0.5cm}
        \begin{itemize}
            \item Fui becado en el \textit{Programa Educacional para Niños, Niñas y Jóvenes con Talentos Académicos} (PROENTA) de la Universidad de La Frontera.
            \item Participé en la \textit{Olimpiada Regional de Matemática} y en la \textit{Olimpiada Regional de Física}, obteniendo en todos los primeros lugares en ambas olimpiadas durante los 4 años que participé. Del mismo modo, participé en otras instancias como \textit{El Juego de Aprender} (Universidad Mayor), la \textit{Olimpiadas de Matemática Aplicada} (INACAP) y los \textit{Juegos Matemáticos Interregionales} (Colegio San Mateo, Osorno), obteniendo los primeros lugares durante todos los años de mi participación, tanto en el modo grupal como en el modo individual.
            \item Originalmente entré a la carrera de Ingeniería Civil Matemática en la Universidad de La Frontera, donde rendí exámenes de suficiencia que me permitieron eximirme del primer semestre del programa. En esa casa de estudios obtuve una beca de intercambio a la Universidad Técnica Federico Santa María y posteriormente ingresé a la Universidad de Chile por admisión especial.
            \item Me gradué con distinción máxima tanto en mi pregrado como en mi postgrado.
            \item Fui ayudante en los cursos de \href{https://www.u-cursos.cl/ingenieria/2024/1/MDS7104/1/datos_curso/bajar_programa?id=95806&1684365667}{Aprendizaje de Máquinas}, \href{https://www.u-cursos.cl/ingenieria/2022/2/CC6204/1/datos_curso/bajar_programa?id=30074&428737004}{Deep Learning}, \href{https://www.u-cursos.cl/ingenieria/2017/1/MA1102/1/datos_curso/bajar_programa?id=3462&17938346}{Álgebra Lineal} y \href{https://www.u-cursos.cl/ingenieria/2017/1/FI1002/2/datos_curso/bajar_programa?id=3453&8748495}{Sistemas Newtonianos} en la Universidad de Chile, además de Cálculo Multivariable y Bases Matemáticas en la Universidad de La Frontera.
            \item He asistido como oyente a cursos de otras universidades y a congresos para ampliar mis conocimientos en diversos temas, generalmente enfocado en tópicos de aprendizaje automático e inteligencia artificial.
            \item Certifiqué mi nivel de inglés con un puntaje B2 en el TOEFL ITP.
        \end{itemize}
    }\\

    \cvbullet{Participación en seminarios y actividades adicionales}{\vspace{-0.5cm}
        \begin{itemize}
            \item Fui voluntario en clases de matemáticas en la penitenciaría de Santiago (Liceo Herbert Vargas Wallis).
            \item He participado en diversos congresos y seminarios de inteligencia artificial como la \textit{Escuela de Verano en Inteligencia Computacional} (EVIC) y la \textit{Jornada Técnica AC3E}, entre otros.
            \item Participé en seminarios para ser optar a un voluntariado en \href{https://technovation.cl/}{Technovation} y poder aportar con mi experiencia y conocimiento a jóvenes que tengan interés en la inteligencia artificial.
        \end{itemize}
    }\\

    \cvbullet{Proyectos personales}{\vspace{-0.5cm}
        \begin{itemize}
            \item He implementado diversos papers y modelos, lo que me ha permitido ir ganando un background robusto en tópicos asociados a la inteligencia artificial.
            \item En paralelo a mis trabajos principales, estoy desarrollando un blog en español sobre inteligencia artificial, con el objetivo de compartir conocimientos técnicos y reducir la barrera idiomática en el campo.
        \end{itemize}
    }\\

\end{tabular}

\vfill
\center

\begin{minipage}{8cm}
    \centering
    \rule{8cm}{1pt}\\
    \textbf{Fernando Esteban Fêtis Riquelme}\\ Ingeniero matemático -- Magister en Ciencia de Datos
\end{minipage}

\vspace{1cm}

\begin{minipage}[t]{12cm}
    \center
\color{black!70}
{\small
\icon{\faMapMarker} Santiago, Chile
\icon{\faPhone} +56 9 4102 2335
\icon{\faAt} \protect\url{fernando.fetis@uchile.cl} \protect\url{https://fernandofetis.notion.site}
}

\end{minipage}

\newpage
\end{paracol}

\end{document}