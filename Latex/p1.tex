\documentclass[lighthipster]{cvtheme}
\usepackage[T1]{fontenc}
\usepackage[utf8]{inputenc}
\usepackage[spanish]{babel}
\usepackage[default]{raleway}
\usepackage[margin=1cm, a4paper]{geometry}

\begin{document}
\thispagestyle{empty}
\section*{}

\simpleheader{Fernando Esteban Fêtis Riquelme}{Magister en Data Science | Ingeniero Matemático}

\subsection*{}
\vspace{4em}

\setlength{\columnsep}{1.5cm}
\columnratio{0.23}[0.75]

\begin{paracol}{2}

\paracolbackgroundoptions
\footnotesize
{\setasidefontcolour
\flushright
\vspace{1cm}
\begin{center}
    \roundpic{images/me.png}
\end{center}

\vspace{0.8cm}
\bg{cvgreen}{white}{Acerca de mí}\\[0.5em]
Soy una persona apasionada por la inteligencia artificial y el aprendizaje continuo. Me motiva encontrar soluciones innovadoras que generen un impacto tangible, y siempre busco nuevos desafíos que me permitan aplicar mis habilidades, aprender y crecer profesionalmente. Hace poco creé una \href{https://fernandofetis.notion.site}{página personal} para compartir algunos tópicos de IA generativa.

\vspace{0.8cm}
\bg{cvgreen}{white}{Objetivos a corto plazo}\\[0.5em]
Este primer semestre estaré enfocado en aprender y profundizar en temas de reinforcement learning y, en particular, su aplicación sobre LLMs y MLLMs. Si me da el tiempo, puede ser un buen momento para comenzar a aprender algunas cosas de quantum ML.

\vspace{0.8cm}
\bg{cvgreen}{white}{Objetivos a mediano plazo} \\[0.5em]
Mi objetivo a mediano plazo es ingresar a un programa de doctorado en el campo de la inteligencia artificial con el fin de poder contribuir activamente al avance de esta disciplina en constante evolución. Estoy entusiasmado por seguir ampliando mis horizontes y ayudar a aplicar la inteligencia artificial para resolver problemas del mundo real.

\vfill
\flushleft{
    \infobubble{\faLanguage}{cvgreen}{white}{Nivel B2 TOEFL ITP}
    \infobubble{\faAt}{cvgreen}{white}{fernando.fetis@uchile.cl}
    \infobubble{\faPhone}{cvgreen}{white}{+56 9 4102 2335}
    
}

}
\switchcolumn\small

\section*{Formación académica}
\vspace{0.5cm}

\begin{tabular}{r| p{0.5\textwidth} c}

    \cvevent{2023--2024}{Magister en Ciencia de Datos}{Universidad de Chile}{fcfm \color{cvred}}{Continué mis estudios entrando a este magíster, el cual está enfocado en procesamiento de datos y uso de herramientas de IA. Me especialicé fuertemente en temas de IA generativa, concluyendo con un \href{https://github.com/fernando-fetis/mds-thesis}{trabajo de tesis} acerca de los puentes de Schrödinger, el cual es un paradigma reciente que generaliza a los modelos de difusión.}{uchile.png} \\

    \cvevent{2017--2023}{Ingeniería Civil Matemática}{Universidad de Chile}{fcfm \color{cvred}}{Me especialicé fuertemente en aprendizaje automático e inteligencia artificial, desarrollando una sólida base en matemáticas aplicadas y algoritmos. Fui admitido por un proceso de selección especial y durante mis estudios participé en diversos proyectos que combinan teoría y práctica.}{uchile.png} \\

    \cvevent{2015--2016}{Ingeniería Civil Matemática}{Universidad de La Frontera}{Campus Bello \color{cvred}}{Inicié mis estudios en esta universidad debido a mi cercana relación con ella. Posteriormente, obtuve una beca que me permitió realizar un intercambio en la Universidad Técnica Federico Santa María, lo que eventualmente llevó a mi traslado definitivo a la Universidad de Chile.}{ufro.png}

\end{tabular}

\section*{Antecedentes laborales}
\vspace{0.5cm}

\begin{tabular}{r| p{0.55\textwidth} c}

    \cvevent{2025}{Profesor curso Modelos Generativos Profundos}{Universidad de Chile}{Santiago \color{cvred}}{En otoño de este año dictaré el curso \href{https://www.u-cursos.cl/ingenieria/2023/2/MDS7203/1/datos_curso/bajar_programa?id=150626&491897610}{Modelos Generativos Profundos}, asociado al programa de Magister en Ciencia de Datos de la Facultad de Ciencias Físicas y Matemáticas.}{uchile.png} \\

    \cvevent{2023}{Búsqueda de arquitecturas neuronales para PINNs}{Inria Chile}{Santiago \color{cvred}}{Posterior a mi práctica profesional, continué mi trabajo en Inria, esta vez enfocado en la optimización de arquitecturas de redes neuronales basadas en la física (PINNs) mediante algoritmos genéticos guiados por procesos gaussianos. El trabajo concluyó con la creación de una librería de Python enfocada en PINNs.}{inria.png}\\

    \cvevent{2023}{Investigación de redes neuronales basadas en la física}{Inria Chile}{Santiago \color{cvred}}{Durante mi segunda y tercera práctica profesional, formé parte de un equipo de investigación enfocado en aplicar redes neuronales a la simulación de sistemas complejos y problemas dinámicos. Contribuí a analizar y mejorar la precisión de estas técnicas, buscando soluciones innovadoras para la resolución de problemas en diversas áreas.}{inria.png}\\

    \cvevent{2021}{Edición del apunte del curso de Machine Learning}{Universidad de Chile}{Santiago \color{cvred}}{Colaboré en la revisión y mejora del material educativo para el curso de \href{https://www.u-cursos.cl/ingenieria/2021/1/MA5204/1/datos_curso/bajar_programa?id=39073&2038924245}{Aprendizaje de Máquinas}, actualizando contenidos y corrigiendo errores. También añadí secciones clave que alineaban mejor el material con los objetivos del programa.}{uchile.png} \\

\end{tabular}

\vfill
\newpage

\end{paracol}

\end{document}