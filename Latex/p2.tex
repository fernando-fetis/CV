\documentclass[lighthipster]{cvtheme}
\usepackage[T1]{fontenc}
\usepackage[utf8]{inputenc}
\usepackage[spanish]{babel}
\usepackage[default]{raleway}
\usepackage[margin=1cm, a4paper]{geometry}

\begin{document}
\thispagestyle{empty}

\setlength{\columnsep}{1.5cm}
\columnratio{0.23}[0.75]



\begin{paracol}{2}

\paracolbackgroundoptions
\footnotesize
{\setasidefontcolour
\flushright

$ $


\vspace{0.8cm}
\bg{cvgreen}{white}{Proyectos personales}\\[0.5em]
\begin{itemize}
    \item He implementado diversos papers y modelos, lo que me ha permitido ir ganando un background robusto en tópicos asociados a la inteligencia artificial.
    \item Actualmente estoy buscando formar un grupo de lectura enfocado en discutir avances recientes en el aprendizaje automático y temas afines.
    \item En paralelo estoy desarrollando un blog en español sobre inteligencia artificial, con el objetivo de compartir conocimientos técnicos y reducir la barrera idiomática en el campo.
\end{itemize}


}
\switchcolumn\small

\section*{Antecedentes laborales}
\vspace{0.5cm}

\begin{tabular}{r| p{0.55\textwidth} c}

    \cvevent{2020}{Simulación de sistema dinámico asociado a la extracción minera}{Centro de Modelamiento Matemático}{Santiago \color{cvred}}{Durante mi primera práctica profesional, desarrollé una interfaz gráfica en MATLAB para visualizar y analizar datos relacionados con la lixiviación in situ en un proyecto minero. Mi trabajo facilitó la interpretación de los datos, mejorando la toma de decisiones en el proceso de optimización operativa.}{cmm.png}\\

    \cvevent{2015}{Asistente en Olimpiadas de Matemáticas}{Universidad de La Frontera}{Temuco \color{cvred}}{Colaboré en la organización de la \textit{Olimpiada Regional de Matemática}, participando en la corrección de pruebas y en la creación de preguntas. Además, contribuí a la edición del libro de problemas de publicación anual.}{ufro.png}

\end{tabular}

\section*{Otros antecedentes}
\vspace{0.5cm}

\begin{tabular}{r p{0.65\textwidth} c}

    \cvbullet{Experiencia académica y docencia}{\vspace{-0.5cm}
        \begin{itemize}
            \item Fui becado en el \textit{Programa Educacional para Niños, Niñas y Jóvenes con Talentos Académicos} (PROENTA) de la Universidad de La Frontera.
            \item Participé en la \textit{Olimpiada Regional de Matemática} y en la \textit{Olimpiada Regional de Física}, obteniendo los primeros lugares en ambas olimpiadas durante los 4 años que participé. Del mismo modo, participé en otras instancias como \textit{El Juego de Aprender} (Universidad Mayor), la \textit{Olimpiadas de Matemática Aplicada} (INACAP) y los \textit{Juegos Matemáticos Interregionales} (Colegio San Mateo, Osorno), obteniendo los primeros lugares durante todos los años de mi participación, tanto en el modo grupal como en el modo individual.
            \item Rendí exámenes de suficiencia que me permitieron eximirme del primer semestre del Plan Común de Ingeniería en la Universidad de La Frontera.
            \item Obtuve una beca de intercambio a la Universidad Técnica Federico Santa María y posteriormente ingresé a la Universidad de Chile por admisión especial.
            \item Fui ayudante en los cursos de \href{https://www.u-cursos.cl/ingenieria/2024/1/MDS7104/1/datos_curso/bajar_programa?id=95806&1684365667}{Aprendizaje de Máquinas}, \href{https://www.u-cursos.cl/ingenieria/2022/2/CC6204/1/datos_curso/bajar_programa?id=30074&428737004}{Deep Learning}, \href{https://www.u-cursos.cl/ingenieria/2017/1/MA1102/1/datos_curso/bajar_programa?id=3462&17938346}{Álgebra Lineal} y \href{https://www.u-cursos.cl/ingenieria/2017/1/FI1002/2/datos_curso/bajar_programa?id=3453&8748495}{Sistemas Newtonianos} en la Universidad de Chile, además de Cálculo Multivariable y Bases Matemáticas en la Universidad de La Frontera.
            \item He asistido como oyente a cursos de otras universidades y a congresos para ampliar mis conocimientos en diversos temas, generalmente enfocado en tópicos de aprendizaje automático e inteligencia artificial.
            \item Certifiqué mi nivel de inglés con un puntaje B2 en el TOEFL ITP.
        \end{itemize}
    }\\

    \cvbullet{Participación en seminarios y actividades adicionales}{\vspace{-0.5cm}
        \begin{itemize}
            \item Recientemente, participé en seminarios para ser optar a un voluntariado en \href{https://technovation.cl/}{Technovation} y poder aportar con mi experiencia y conocimiento a jóvenes que tengan interés en la inteligencia artificial.
            \item Fui voluntario en clases de matemáticas en la penitenciaría de Santiago (Liceo Herbert Vargas Wallis).
            \item He participado en diversos congresos y seminarios de inteligencia artificial como la \textit{Escuela de Verano en Inteligencia Computacional} (EVIC) y la \textit{Jornada Técnica AC3E}, entre otros.
        \end{itemize}
    }\\



\end{tabular}


\vfill
\center

% Firma:
\begin{minipage}{7cm}
    \centering
    \rule{7cm}{1pt}\\
    \textbf{Fernando Esteban Fêtis Riquelme}\\ Magister en Ciencia de Datos
\end{minipage}

\vspace{1.5cm}

% Pie de página:
\begin{minipage}[t]{12cm}
    \center
\color{black!70}
{\small
\icon{\faMapMarker} Santiago, Chile
\icon{\faPhone} +56 9 4102 2335
\icon{\faAt} \protect\url{fernando.fetis@uchile.cl} \protect\url{https://fernandofetis.notion.site}
}


\end{minipage}

\newpage
\end{paracol}



\end{document}